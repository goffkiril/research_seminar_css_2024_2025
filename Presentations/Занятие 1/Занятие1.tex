% Copyright 2004 by Till Tantau <tantau@users.sourceforge.net>.
%
% In principle, this file can be redistributed and/or modified under
% the terms of the GNU Public License, version 2.
%
% However, this file is supposed to be a template to be modified
% for your own needs. For this reason, if you use this file as a
% template and not specifically distribute it as part of a another
% package/program, I grant the extra permission to freely copy and
% modify this file as you see fit and even to delete this copyright
% notice. 

\documentclass{beamer}

% There are many different themes available for Beamer. A comprehensive
% list with examples is given here:
% http://deic.uab.es/~iblanes/beamer_gallery/index_by_theme.html
% You can uncomment the themes below if you would like to use a different
% one:
%\usetheme{AnnArbor}
%\usetheme{Antibes}
%\usetheme{Bergen}
%\usetheme{Berkeley}
%\usetheme{Berlin}
%\usetheme{Boadilla}
%\usetheme{boxes}
%\usetheme{CambridgeUS}
%\usetheme{Copenhagen}
%\usetheme{Darmstadt}
\usetheme{default}
%\usetheme{Frankfurt}
%\usetheme{Goettingen}
%\usetheme{Hannover}
%\usetheme{Ilmenau}
%\usetheme{JuanLesPins}
%\usetheme{Luebeck}
%\usetheme{Madrid}
%\usetheme{Malmoe}
%\usetheme{Marburg}
%\usetheme{Montpellier}
%\usetheme{PaloAlto}
%\usetheme{Pittsburgh}
%\usetheme{Rochester}
%\usetheme{Singapore}
%\usetheme{Szeged}
%\usetheme{Warsaw}
\newenvironment{trienv}{\only{\setbeamertemplate{items}[triangle]}}{}
\usepackage{wrapfig}
\usepackage{multimedia}
\usepackage{multirow}
\usepackage[round]{natbib}
\usepackage[utf8]{inputenc}
\usepackage[russian]{babel}
\usepackage{graphicx} 
\usepackage{tikz}
\usepackage{bibentry}
\useoutertheme{miniframes}
\setbeamercolor{frametitle}{fg = black}
\setbeamercolor{title}{fg = black}
\defbeamertemplate{itemize item}{mybullet}{\large\raise1pt\hbox{\textbullet}}
\setbeamertemplate{itemize item}[mybullet]
\setbeamertemplate{navigation symbols}{}
%\bibliographystyle{apsr_fs} 
    \defbeamertemplate*{footline}{my infolines theme}
    {
       \leavevmode%
       \hbox{%
       \begin{beamercolorbox}[wd=.333333\paperwidth,ht=2.25ex,dp=1ex,center]{author in head/foot}%
         \usebeamerfont{author in head/foot}\insertshortauthor~~\insertshortinstitute
       \end{beamercolorbox}%
       \begin{beamercolorbox}[wd=.333333\paperwidth,ht=2.25ex,dp=1ex,center]{title in head/foot}%
         \usebeamerfont{title in head/foot}\insertshorttitle
       \end{beamercolorbox}%
       \begin{beamercolorbox}[wd=.333333\paperwidth,ht=2.25ex,dp=1ex,right]{date in head/foot}%
         \usebeamerfont{date in head/foot}\insertshortdate{}\hspace*{2em}
         \insertframenumber{} / \inserttotalframenumber\hspace*{2ex} 
       \end{beamercolorbox}}%
       \vskip0pt%
    }
\setbeamerfont{frametitle}{size=\small}
\title[]{\textbf{\Large{
Научный Подход в Современных Социальных Науках.}}}
\author[]{%
  \texorpdfstring{%
    \begin{columns}
      \column{.3333\linewidth}
      \centering
      \large{}
      \column{.40\linewidth}
      \centering
      \large{Евгений Седашов, PhD} \\
      \small{esedashov@hse.ru}
      \column{.3333\linewidth}
      \centering
      \large{}
    \end{columns}
 }
 {Author 1, Author 2, Author 3}
}


% - Give the names in the same order as the appear in the paper.
% - Use the \inst{?} command only if the authors have different
%   affiliation.


% - Use the \inst command only if there are several affiliations.
% - Keep it simple, no one is interested in your street address.

\date{6/11/2024}
% - Either use conference name or its abbreviation.
% - Not really informative to the audience, more for people (including
%   yourself) who are reading the slides online

\subject{}
\usecolortheme{default}
% This is only inserted into the PDF information catalog. Can be left
% out. 

% If you have a file called "university-logo-filename.xxx", where xxx
% is a graphic format that can be processed by latex or pdflatex,
% resp., then you can add a logo as follows:

% \pgfdeclareimage[height=0.5cm]{university-logo}{university-logo-filename}
% \logo{\pgfuseimage{university-logo}}

% Delete this, if you do not want the table of contents to pop up at
% the beginning of each subsection:
\begin{document} 
\makeatletter
\let\beamer@old@writeslidentry\beamer@writeslidentry
\newcommand\bulletoff{\let\beamer@writeslidentry\relax}
\newcommand\bulleton{\let\beamer@writeslidentry\beamer@old@writeslidentry}
%------------------------------------------------
\large
\setbeamercovered{transparent}
\begin{frame}
  \titlepage
\end{frame}
\begin{frame}
  \includegraphics[scale=1]{qrcode.png}
\end{frame}
% Section and subsections will appear in the presentation overview
% and table of contents.
\section{Введение}
\begin{frame}{Цели Курса}
\begin{itemize}
	\setlength\itemsep{2em}
	\item Данный курс можно рассматривать как введение в методологию современных социально-научных исследований.
	\item Первая цель -- понять,  как мы можем получить научные результаты,  опираясь на анализ данных. 
	\item Вторая цель -- облегчить выполнение курсовой работы. 
\end{itemize}
\end{frame}
\begin{frame}{Оценивание}
\begin{itemize}
	\setlength\itemsep{2em}
	\item Домашние задания -- 40 процентов.
	\item Посещение и активность -- 20 процентов.
	\item Финальный текст -- 40 процентов. 
\end{itemize}
\end{frame}
\section{Научный Подход}
\begin{frame}{Научный Подход vs.  Здравый Смысл}
\begin{itemize}
	\setlength\itemsep{2em}
	\item Наука, в наиболее общем смысле, ставит целью получение знаний об окружающем нас мире. 
	\item Наука -- лишь один из способов получения подобных знаний.
	\item Другие способы -- игра,  метод проб и ошибок,  здравый смысл (часто основанный на опыте). 
	\item В чём заключаются отличительные особенности  научного подхода? 
\end{itemize}
\end{frame}
\begin{frame}{Принципы Научного Подхода}
\begin{itemize}
	\setlength\itemsep{2em}
	\item Научные исследования опираются на системные теории,  описывающие те или иные аспекты окружающего мира.
	\item Научные исследования подвергают теории тщательной эмпирической проверке.  
	\item Цель научных исследований -- в выявлении взаимосвязей между переменными. 
	\item Эмпирическая тестируемость выдвигаемых предположений (критерий фальсификации).
\end{itemize}
\end{frame}
\begin{frame}{Критерий Фальсификации I}
\begin{itemize}
	\setlength\itemsep{2em}
	\item Фальсифицируемость выдвигаемых предположений означает принципальную возможность их эмпирического опровержения. 
	\item \textbf{Пример 1}.  Положительное подкрепление в процессе обучения ведёт к более высоким результатам контрольных работ,  чем отрицательное подкрепление.  
	\item \textbf{Пример 2}.  Люди с более высоким уровнем образования и дохода чаще приходят на выборы.  
\end{itemize}
\end{frame}
\begin{frame}{Критерий Фальсификации II}
\begin{itemize}
	\setlength\itemsep{2em}
	\item \textbf{Пример 3}.  Футболисты,  имевшие опыт жизни в странах с активными гражданскими конфликтами,  будут более склонны к агрессивной игре. 
	\item \textbf{Пример 4}.  Земля -- единственная планета с разумной формой жизни.
	\item Все 4 примера описывают утверждения,  которые можно эмпирически опровергнуть. 
	\item Ваши примеры? 
\end{itemize}
\end{frame}
\begin{frame}{Границы Научного Метода}
\begin{itemize}
	\setlength\itemsep{2em}
	\item \textbf{Ловушка Тавтологии} -- тавтологические утверждения не являются научными (они верны по определению).  
	\item \textbf{Ненаблюдаемые явления} -- противоречат критерию фальсификации и поэтому лежат за рамками научного метода. 
\end{itemize}
\end{frame}
\begin{frame}{Четыре Метода Познания I}
\begin{itemize}
	\setlength\itemsep{2em}
	\item Чарльз Пирс определяет четыре метода познания.
	\item Метод упорства -- человек твёрдо верит в определённые факты,  потому что  ``это так работает'', ``так было всегда'',  ``у меня опыт''.  Факты,  противоречащие устоявшейся картине мира,  отбрасываются как ненужные. 
	\item Метод авторитета -- источником знаний является авторитет в той или иной форме.  Советы доктора или юриста относятся к данному методу,  как и любое доверие к экспертной точке зрения (``доверяю,  потому что он специалист''). 
\end{itemize}
\end{frame}
\begin{frame}{Четыре Метода Познания II}
\begin{itemize}
	\setlength\itemsep{2em}
	\item Метод а-приори -- знание,  условно известное заранее,  до получения практического опыта. 
	\item Научный метод -- метод получения знаний о мире,  идеалом которого является объективность и реплицируемость; любой исследователь,  имея на руках одни и те же вводные,  должен получить одни и те же результаты.  
	\item Реплицируемость $\neq$ детерминированность! 
\end{itemize}
\end{frame}
\section{Проблемы и Гипотезы}
\begin{frame}{Проблема Исследования}
\begin{itemize}
	\setlength\itemsep{2em}
	\item Любое исследование начинается с постановки научной проблемы; другим названием также является исследовательский вопрос.
	\item Формулировка проблемы для прикладных проектов имеет свою специфику; например,  проблемой может быть отсутствие системы навигации для маломобильных граждан (а целью -- создание такой системы). 
	\item Единого рецепта формулировки научной проблемы нет,  но есть общие гайдлайны,  которыми можно руководствоваться. 
\end{itemize}
\end{frame}
\begin{frame}{Примеры I}
\begin{itemize}
	\setlength\itemsep{2em}
	\item Как экономическое неравенство связано с политическими транзитами? 
	\item Как разные типы стимулирования учеников влияют на образовательные результаты? 
	\item Как звуковая среда влияет на восприятие визуальной информации? 
\end{itemize}
\end{frame}
\begin{frame}{Примеры II}
\begin{itemize}
	\setlength\itemsep{1em}
	\item Как личностные особенности лидеров влияют на конфликты между странами? 
	\item Как фрустрация влияет на уровень агрессии? 
	\item Влияет ли расовая принадлежность подозреваемого на вероятность обвинительного приговора в суде? 
	\item В качестве проблемы,  таким образом,  обычно выступает вопрос:  ``Как связаны две (или более) переменные?''
	\item Исключение -- таксономические исследования,  ставящие целью разработку классификации,  и методологические исследования.
\end{itemize}
\end{frame}
\begin{frame}{Критерии Качественной Формулировки Проблемы}
\begin{itemize}
	\setlength\itemsep{2em}
	\item В формулировке всегда должно быть указание на взаимосвязь между переменными. 
	\item Сформулируйте вопрос,  как в примерах выше.
	\item Возможность эмпирической проверки. 
\end{itemize}
\end{frame}
\begin{frame}{Гипотезы}
\begin{itemize}
	\setlength\itemsep{1em}
	\item Гипотезы являются эмпирически проверяемыми утверждениями о связи между переменными. 
	\item Гипотезы не возникают ``из воздуха'' -- за ними обычно стоят серьёзные теоретические конструкции. 
	\item Принципиальное отличие гипотезы от проблемы -- в гипотезах есть указание на \textbf{направление} связи,  в то время как в проблеме есть только указание на \textbf{возможное наличие} связи. 
\end{itemize}
\end{frame}
\begin{frame}{Критерии Хорошей Гипотезы I}
\begin{itemize}
	\setlength\itemsep{3em}
	\item Указание на направление связи между переменными.
	\item Явные эмпирически проверяемые следствия. 
	\item Использование дескриптивного (``увеличивает'',  ``уменьшает'') вместо нормативного (``улучшает'',  ``ухудшает'') языка. 
\end{itemize}
\end{frame}
\begin{frame}{Критерии Хорошей Гипотезы II}
\begin{itemize}
	\setlength\itemsep{3em}
	\item Насколько обобщённой должна быть гипотеза? 
	\item Размер букв увеличивает скорость чтения -- слишком общо. 
	\item Размер шрифта Helvetica увеличивает скорость чтения и количество запоминаемой информации -- в самый раз. 
\end{itemize}
\end{frame}
\begin{frame}{Пример Гипотез}
\begin{itemize}
	\setlength\itemsep{3em}
	\item Уровень шума или музыки ухудшает качестве визуального восприятия. 
	\item Неравенство влияет на переход от недемократических к демократическим режимам в форме параболы с ветвями вниз (inversed U-shaped). 
	\item Количество участвующих повстанческих группировок увеличивает ожидаемую продолжительность гражданских войн. 
\end{itemize}
\end{frame}
\begin{frame}{Откуда Берутся Гипотезы?}
\begin{itemize}
	\setlength\itemsep{1em}
	\item Применить существующую теоретическую модель к вопросу,  который ранее не анализировался с её помощью. 
	\item Модифицировать теорию,  которая оказалась ложной. 
	\item Придумать новую теорию. 
\end{itemize}
\end{frame}
\begin{frame}{Пример Модификации Теории I}
\begin{itemize}
	\setlength\itemsep{1em}
	\item Рассмотрим известную модель голосования Даунса.
	\item Даунс (1957) предложил простую модель голосования: $U(V)=P\cdot B - C$.  Правило принятия решения следующее: голосуй, если $U(V) \geqslant 0$, в противном случае оставайся дома; $P$ --вероятность повлиять на исход выборов -- практически всегда очень маленькая,  поэтому парадоксальным образом данная модель предсказывает нулевую явку. 
	\item Возможное решение -- добавить в модель гражданский долг: $U(V)=P\cdot B - C + D$ (модель Райкера-Ордешока). 
\end{itemize}
\end{frame}
\begin{frame}{Задание}
\begin{itemize}
	\setlength\itemsep{1em}
	\item Какие переменные могут влиять на академические достижения школьников? Сформулируйте теорию и тестируемые гипотезы. 
	\item Какие переменные могут влиять на отношение к трудовым мигрантам? Сформулируйте теорию и тестируемые гипотезы. 
\end{itemize}
\end{frame}
\section{Конструкты и Операционализация}
\begin{frame}{Концепты и Конструкты}
\begin{itemize}
	\setlength\itemsep{1em}
	\item Концепт -- абстрактное понятие,  обозначающее класс схожих по характеристикам феноменов-объектов.
	\item Вес,  масса,  энергия,  сила,  эмоция,  политический режим -- всё это примеры концептов. 
	 \item Конструкт -- это тип концепта,  специально созданный для научных целей.
\end{itemize}
\end{frame}
\begin{frame}{Характеристики Конструктов}
\begin{itemize}
	\setlength\itemsep{1em}
	\item Конструкты участвуют в построении теорий.
	\item Конструкты определяются таким образом,  чтобы их можно было наблюдать и измерять; концепты, в отличие от конструктов,  не обязательно являются измеряемыми понятиями. 
\end{itemize}
\end{frame}
\begin{frame}{Переменные}
\begin{itemize}
	\setlength\itemsep{1em}
	\item Переменная -- это измеряемая величина,  созданная с целью проверки конкретной гипотезы. 
	\item Примеры переменных: пол,  возраст,  доход,  уровень образования,  балл на образовательном тесте,  визуальное восприятие, эмоциональное состояние. 
	\item С математической точки зрения, мы, как правило, имеем дело с бинарными,  категориальными или непрерывными переменными. 
\end{itemize}
\end{frame}
\begin{frame}{Определения Конструктов и Переменных}
\begin{itemize}
	\setlength\itemsep{2em}
	\item Содержательное определение конструкта -- это определение,  использующее другие конструкты. 
	\item Операционное определение конструкта -- это определение,  задающее параметры его измерения. 
	\item Операционное определение интеллекта -- балл, набранный в специально разработанном тесте.
	\item Операционное определение неравенства -- разница между 10 процентами самых богатых и 10 процентами самых бедных.
\end{itemize}
\end{frame}
\begin{frame}{Типы Операционных Определений}
\begin{itemize}
	\setlength\itemsep{2em}
	\item Измеряемое определение -- описывает,  как будет измеряться конкретная переменная; пример -- отношение к трудовым мигрантам можно измерять вопросом  ``Как бы Вы отнеслись к факту соседства с Вами семьи мигрантов? Оцените Ваше отношение по 5 балльной шкале, где 1 -- определённо отрицательно,  5 -- определённо положительно''.  
	\item Экспериментальное определение -- описывает то,  как исследователь может манипулировать значениями переменной в контролируемой форме.  
\end{itemize}
\end{frame}
\begin{frame}{Задание -- Операционные Определения}
\begin{itemize}
	\setlength\itemsep{2em}
	\item Лидерство.
	\item Популярность.
	\item Дискриминация. 
	\item Политическая идеология. 
	\item Переменные из ваших курсовых. 
\end{itemize}
\end{frame}
\end{document}
