\documentclass[12pt]{article}
\usepackage[margin=1in]{geometry}
\usepackage{setspace,float,amsmath,amsxtra,hyperref,authblk,amssymb,graphicx,multirow,xcolor,lscape,rotating,url,fancyhdr,multirow,tabularx,booktabs}
\usepackage[utf8]{inputenc}
\usepackage[russian]{babel}
\usepackage{natbib}
\usepackage{bibentry}
\usepackage{indentfirst}
\bibliographystyle{plain}
\pagestyle{fancy}
\newcommand\tab[1][1cm]{\hspace*{#1}}
\lhead{Методология и Методы Исследований в Социальных Науках}
\rhead{Седашов, 2024}
\hypersetup {
colorlinks=true,
linkcolor=cyan,
urlcolor=blue,
filecolor=magenta
}
\linespread{1.3}
\begin{document}
\begin{center}
\huge \textbf{Научно-Исследовательский Семинар  ``Методология и Методы Исследований в Социальных Науках''}
\end{center}
\vspace{5 mm}

\textbf{Преподаватель:} \href{https://www.hse.ru/staff/sedashov}{Evgeny Sedashov} 
\vspace{2 mm}

\textbf{E-mail:} \href{mailto:esedashov@hse.ru}{esedashov@hse.ru}
\vspace{2 mm}

\textbf{Время занятий:} Среда, 13:40 -- 17:40 (с 20 мин. перерывом). 
\vspace{2 mm}

\textbf{Часы консультаций:} Индивидуальные консультации назначаются по e-mail. 
\vspace{2 mm}

\textbf{Ассистенты:} \href{https://www.hse.ru/staff/KirilHSE/}{Григорий Кирюхов}, София Янис
\vspace{2 mm}


\section*{Описание Курса} 
Современные методы анализа данных прочно вошли в исследовательский репертуар социальных наук. Данный курс преследует несколько целей. Во-первых, его можно рассматривать как введение в методологию количественных исследований. Мы рассмотрим ряд наиболее важных аналитических инструментов, которые часто используются в современных исследованиях. Будут затронуты следующие темы: каузальный анализ, экспериментальные и квази-экспериментальные исследовательские дизайны, а также базовые инструменты регрессионного анализа. Во-вторых, курс ставит целью подготовку студентами полноценного научного исследования, включающего постановку исследовательского вопроса, разработку исследовательского дизайна, поиск и разбор релевантной литературы, сбор и анализ данных. Конечным результатом должен быть научный текст (см. ниже), который студенты должны сдать в финале курса.  
\section*{Пререквизиты} 
Формальных пререквизитов для курса нет. 
\section*{Программное Обеспечение} 
Python 3,  библиотеки numpy,  pandas,  statsmodels 
\section*{Оценивание}
\begin{itemize}
\item Все задания оцениваются по 10-балльной системе. В конце курса я рассчитаю общую оценку на основе веса каждого задания.
\item Оценка рассчитывается по следующему правилу: Домашние Задания -- 40 \%,  Посещение и Активность -- 20 \%,  Финальный Текст -- 40 \%.  
\end{itemize}
\section*{Примерный Календарь Курса}
\begin{table}[H]
\begin{tabular}{p{0.8\textwidth}@{}p{0.4\textwidth}@{}}
Научный подход в современных социальных науках \dotfill & Ноябрь, 2 \\ 
Исследовательский дизайн: введение \dotfill & Ноябрь, 9 \\
Автоматизированный сбор данных: парсинг, скрейпинг \dotfill & Ноябрь, 16 \\
Обработка данных. Мерджинг, решейпинг. Архитектуры баз данных \dotfill & Ноябрь, 23 \\
Опросные эксперименты: основные типы и проблемы \dotfill & Ноябрь, 30 \\
Квазиэкспериментальные исследовательские дизайны \dotfill & Декабрь, 7 \\
Частотный анализ. Гипотезы. ANOVA \dotfill & Декабрь, 14 \\
Парная линейная регрессия  \dotfill & Январь, 18 \\
Множественная линейная регрессия \dotfill & Январь, 25 \\
Бинарные зависимые переменные \dotfill & Февраль, 1 \\
Счётные зависимые переменные \dotfill & Февраль, 8 \\
Кластерный и факторный анализ \dotfill & Февраль, 15 \\
Мэтчинг и инструментальные переменные \dotfill & Февраль, 22 \\
Защита финальных проектов \dotfill & Март, 1 \\

Прерыв \dotfill \\





Множества и связи.  Вариация и Ковариация.\dotfill & Ноябрь, 10 \\ 
Вероятностный подход. Генеральная и выборочная совокупность. \dotfill & Ноябрь, 17 \\
Принципы анализа и интерпретации.  Частотный анализ. Гипотезы.\dotfill & Ноябрь, 24 \\
ANOVA. \dotfill & Декабрь, 1-8 \\
Исследовательский дизайн: введение. \dotfill & Декабрь, 15 \\
Рандомизированные эксперименты. \dotfill & Январь, 20 \\
Квазиэкспериментальные исследовательские дизайны. \dotfill & Январь, 27 \\
Лабораторные,  полевые и опросные эксперименты. \dotfill & Февраль, 3 \\
Неэкспериментальные (observational) исследования. \dotfill & Февраль, 10 \\
Измерения в социальных науках. Валидность и надежность. \dotfill & Февраль, 17 \\
Методы наблюдения и сбора данных.  \dotfill & Март, 2 \\
Линейная регрессия и её приложения в социальных науках.  \dotfill & Март, 16 \\
Факторный анализ и каузальные графы.  \dotfill & Март,  23 \\
Заключение.  \dotfill & Март, 23 \\
\end{tabular}
\end{table}
\section*{Литература}
Курс примерно следует в логике следующей книги:

Kerlinger,  Fred N.,  and Howard B.  Lee.  2000.  \textit{Foundations of Behavioral Research. Fourth Edition.} Harcourt College Publishers. 

Также мы будем периодически использовать следующую книгу:

Angrist,  Joshua D. ,  and J\"orn  Steven Pischke.  2009.  \textit{Mostly Harmless Econometrics: An Empiricist's Companion.} Princeton University Press. 
\section*{Разбивка по Параграфам -- Kerlinger and Lee}
\begin{table}[H]
\begin{tabular}{p{0.8\textwidth}@{}p{0.4\textwidth}@{}}
Научный подход в современных социальных науках. \dotfill & Ch. 1 -- 3 \\ 
Множества и связи.  Вариация и Ковариация.\dotfill & Ch. 4 --6 \\ 
Вероятностный подход. Генеральная и выборочная совокупность. \dotfill & Ch. 7 -- 8 \\
Принципы анализа и интерпретации.  Частотный анализ. Гипотезы.\dotfill & Ch. 9 -- 12 \\
ANOVA. \dotfill & Ch.  13 -- 16 \\
Исследовательский дизайн: введение. \dotfill & Ch. 18 -- 20 \\
Рандомизированные эксперименты. \dotfill & Ch. 21 \\
Квазиэкспериментальные исследовательские дизайны. \dotfill & Ch. 22 \\
Лабораторные,  полевые и опросные эксперименты. \dotfill & Ch.  24 -- 25 \\
Неэкспериментальные (observational) исследования. \dotfill & Ch.  23 \\
Измерения в социальных науках. Валидность и надежность. \dotfill & Ch. 26 -- 28 \\
Методы наблюдения и сбора данных.  \dotfill & Ch. 29 -- 31 \\
Линейная регрессия и её приложения в социальных науках.  \dotfill & Ch. 32 -- 33 \\
Факторный анализ и каузальные графы.  \dotfill & Ch. 34 -- 35 \\
Заключение.  \dotfill & TBA \\
\end{tabular}
\end{table}
\section*{Финальный Текст}
Есть два варианта финального текста:

1) mock report (программа минимум) -- полноценная статья с описанием исследовательского дизайна,  но без анализа данных; другими словами,  это должен быть полноценный текст статьи с описанием результатов, как если бы они подтверждали ваши гипотезы. 

2) драфт курсовой (программа максимум) -- полноценная статья,  с описанием исследовательского дизайна и реальными эмпирическими результатами.  

Оценки за mock report не будут а-приори ниже,  чем за драфт курсовой.  Драфт курсовой -- это,  скорее,  возможность для вас подготовить работу,  равномерно распределяя силы, а не делать всё в последний момент,  как это,  к сожалению, нередко бывает. 
\end{document}
